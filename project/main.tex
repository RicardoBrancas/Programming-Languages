\documentclass{beamer}
\usepackage{color}
\usepackage{graphicx}
\usepackage{tikz}

\usetikzlibrary{arrows,positioning}

\usetheme{fibeamer}

%% ===   DEFINITIONS   ===

\theoremstyle{definition} % See Lesson Three of the LaTeX Manual for more on this kind of "proclamation."
\newtheorem*{dfn}{A Reasonable Definition}

\tikzstyle{box1} = [rectangle, rounded corners, minimum width=3cm, minimum height=1cm, text centered, draw=fibeamer@blue, fill=fibeamer@lightBlue]

%% === END DEFINITIONS ===

\title{Program Synthesis}
\subtitle{An onverview of logic based approaches}
\author{Ricardo Brancas}
\institute{Instituto Superior Técnico}

\AtBeginSection[]{
\begin{frame}
    \frametitle{Presentation Outline}
    \tableofcontents[currentsection]
\end{frame}
}

\begin{document}

\begin{frame}
    \titlepage
\end{frame}

\section{What is it?}

\begin{frame}
    \frametitle{Program Synthesis}

    \centering
    \begin{tikzpicture}[node distance=2cm]

        \node[box1] (spec)                     {Specification};
      \onslide<2->{
        \node[box1] (synth) [below of = spec]  {Synthesiser};
      }
        \node[box1] (prog)  [below of = synth] {Program};

      \onslide<2->{
        \draw[->] (spec)  -- (synth);
        \draw[->] (synth) -- (prog);
      }

    \end{tikzpicture}

    %TODO add exemples of specifications
\end{frame}

\begin{frame}
    \frametitle{A Counterexample}

    Consider the function

    $$ f(x)=
    \begin{cases} % See Lesson Twelve of the LaTeX Manual for more on cases.
    x^2\sin(1/x), &\text{if }x\neq0 \\
    0, &\text{if }x=0
    \end{cases}
    $$

    Let's see what $f'(0)$ is.
\end{frame}

\begin{frame}
    \frametitle{Finding $f'(0)$}

    By the definition of derivative,
    \begin{eqnarray*} % As usual, the asterisk suppresses the numbering of each line in the array.
    f'(0)&=&\pause\displaystyle\lim_{h\to 0}\frac{f(0+h)-f(0)}{h}\\
    \pause&=&\displaystyle\lim_{h\to 0}\frac{h^2\sin(1/h)-0}{h}\\
    \pause&=&\displaystyle\lim_{h\to 0}h\sin(1/h)\\
    \end{eqnarray*}

    Since $-h\leq h\sin(1/h)\leq h$ \pause and $\displaystyle\lim_{h\to 0}(-h)=\displaystyle\lim_{h\to 0}(h)=0$, \pause the \uncover<7->{Squeeze }Theorem says \pause $f'(0)=0.$
    % The command \uncover<m->{STUFF} means that STUFF will appear starting in the mth slide of the frame.
    % The command \uncover<m-n>{STUFF} means that STUFF will appear from the mth slide to the nth slide of the frame.
\end{frame}

\begin{frame}
    \frametitle{What Really Happens at $x=0$?}
    \begin{columns} % This creates a frame with multiple columns.
    \begin{column}{0.5\textwidth} % The first column will be 50% as wide as the width of text on the page.
    But $f(x)$ oscillates wildly as $x\to 0$, so even though $f'(0)=0$, $f$ has neither max, min, nor inflection point at $x=0$.
    \end{column}

    \pause

    \begin{column}{0.5\textwidth} % Now begins our second column.
    \includegraphics[width=5cm, height=5cm]{graph1.png} % Beamer doesn't like to display .eps files. This .png was converted from .eps using Adobe Acrobat. The file graph1.png should be in the same folder as the .tex file.
    \begin{center}
    \textcolor{orange}{$y=f(x)$}, \textcolor{red}{$y=x^2$}, \textcolor{green}{$y=-x^2$} % This changes the text color.
    \end{center}
    \end{column}
    \end{columns}
\end{frame}

\section{What Does $g'(c)>0$ Mean?}

\begin{frame}
    \frametitle{How to Define ``Increasing at a Point''?}
    It's natural to think that if $g'(c)>0$ then $g$ must be ``increasing at $x=c$.''

    \pause But what does ``increasing at $x=c$'' really mean?

    \pause \begin{dfn} % We created the proclamation dfn near the start of the document.
    A function $g$ is \emph{increasing at $x=c$} if there is an open interval $I=(c-\delta,c+\delta)$ such that \pause if $x_1, x_2\in I$, \pause then $x_1<x_2\Rightarrow \pause g(x_1)<g(x_2)$.
    \end{dfn}
\end{frame}

\begin{frame}
    \frametitle{Our Function with a Slight Twist}
    Let's modify our function to

    $$ g(x)=
    \begin{cases}
    0.5x+x^2\sin(1/x), &\text{if }x\neq0 \\
    0, &\text{if }x=0
    \end{cases}
    $$

    Using the definition of derivative as before, we will find that $g'(0)=0.5$.
\end{frame}

\begin{frame}
    \frametitle{What Really Happens at $x=0$?}
    However, $g(x)$ still oscillates enough as $x\to 0$ that there is no open interval containing $x=0$ that satisfies our definition of $g$ increasing at $x=0$ even though $g'(0)>0$.\pause

    \begin{center}
    \includegraphics[width=5cm, height=5cm]{graph2.png}

    \textcolor{orange}{$y=g(x)$}, \textcolor{red}{$y=x^2+0.5x$}, \textcolor{green}{$y=x^2-0.5x$}
    \end{center}

\end{frame}

\section{Further Work}

\begin{frame}
    The function $f(x)$ introduced earlier has other interesting properties, one of which is the fact that while $f'(0)$ exists, $f'(x)$ is discontinuous at $x=0$.\vspace{.5cm}

    We leave it to you to work this out for yourself and to explore this interesting function further.\vspace{.5cm}

    Thank you for your attention today.
\end{frame}

\end{document}
